\documentclass{article}

% set font encoding for PDFLaTeX, XeLaTeX, or LuaTeX
\usepackage{ifxetex,ifluatex}
\if\ifxetex T\else\ifluatex T\else F\fi\fi T%
  \usepackage{fontspec}
\else
  \usepackage[T1]{fontenc}
  \usepackage[utf8]{inputenc}
  \usepackage{lmodern}
\fi

\usepackage{amsmath}
\usepackage{amssymb}
\usepackage{amsthm}

\usepackage{hyperref}
\usepackage{graphicx}

\title{MA320 (Clontz) Writing Portfolio}
\author{Your name here}


\begin{document}
\maketitle

\section{Writing Samples}
\textit{Include five examples of mathematical writing
from this semester. These should include three quiz submissions
that involved non-trivial mathematical writing and problem-solving 
(generally, the question asked to ``explain'' something; see also
the rubric below), one question posted to
\href{https://math.stackexchange.com/}{Math.StackExchange},
and one answer posted to Math.SE.
For each Math.SE post, include
a citation to the post like \cite{1089984},
and be sure to make any necessary adjustments to
typeset your work in LaTeX. You may answer your own question;
if you don't, be sure to provide the original question for your
answer.}

\textit{These posts should reflect
\href{https://usaonline.southalabama.edu/courses/35295/files/6197333?wrap=1}
{Prof. Su's Guidelines for Good Mathematical Writing}
and (for the question/answer) adhere to Math.SE's policies for
\href{https://math.stackexchange.com/help/asking}
{Asking}
and
\href{https://math.stackexchange.com/help/answering}
{Answering}.}

\textit{For each sample, also provide a few sentences reflecting on it.
This should consider any instructor comments on your quiz,
or scores/comments/answers received on your post.
Ultimately for a quiz, did you successfully communicate a solution
that makes sense to a general mathematical audience (not just yourself
or the instructor), without resorting to just minimally altering an
example solution from the instructor? And ultimately for a forum post,
was your question well-posed for a general mathematical audience,
or your answer useful?}

% replace AB1 with the appropriate quiz
\subsection{Quiz Sample \#1 (AB1)}
\subsubsection*{Question}
TODO

\subsubsection*{Answer}
TODO

\subsubsection*{Reflection}
TODO


% replace AB1 with the appropriate quiz
\subsection{Quiz Sample \#2 (AB1)}
\subsubsection*{Question}
TODO

\subsubsection*{Answer}
TODO

\subsubsection*{Reflection}
TODO


% replace AB1 with the appropriate quiz
\subsection{Quiz Sample \#3 (AB1)}
\subsubsection*{Question}
TODO

\subsubsection*{Answer}
TODO

\subsubsection*{Reflection}
TODO





\subsection{Math.SE Question}
\subsubsection*{Question}
TODO \cite{1089984}

\subsubsection*{Reflection}
TODO




\subsection{Math.SE Answer}
\subsubsection*{Question}
% just say "same as above" if you answered your own question
TODO

\subsubsection*{Answer}
TODO \cite{1089984}

\subsubsection*{Reflection}
TODO





\section{Rubric}
\textit{Self-report your progress by marking the box next to
each of the following requirements that you've [M] met
for this portfolio or [S] show signficiant progress towards.}

\begin{itemize}
\item[[ ]] \textbf{Scope.} This portfolio includes exactly
five writing samples: three quizzes, one Math.SE question,
and one Math.SE answer.
\item[[ ]] \textbf{Mathematics.} This portfolio engages in
mathematical concepts appropriate for a student enrolled in
MA320. In particular, these samples are generally concerned
with understanding and proving mathematical facts, not just
computing values.
\item[[ ]] \textbf{Reflection.} This portfolio includes
appropriate reflections for each sample. Each sample is observed
to appropriately communicate a mathematical idea, question, or answer.
Quiz answers are the students' original writing, and not lightly
modified solutions from the instructor.
\item[[ ]] \LaTeX{}. This portfolio was properly typeset
using \LaTeX{}. In particular, mathmode (\verb|\(x^2\)|) was
used appropriately for mathematical expressions (\(x^2\))
in each post, and BibTeX \cite{bibtexDocs}
was used for reference management.
\end{itemize}


\section{Overall Quality}
\textit{Self-report your progress by marking the box [X] next to
which of the following best describes the overall quality of
this portfolio.}

\begin{itemize}
\item[[ ]] \textbf{Shows Progress}: This portfolio at least demonstrates
significant progress towards each of the rubric items.

\item[[ ]] \textbf{Meets Expectations}: This portfolio meets
each of the rubric items.

\end{itemize}


\bibliographystyle{plain}
\bibliography{references}

\end{document}
