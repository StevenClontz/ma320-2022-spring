\documentclass{article}

% set font encoding for PDFLaTeX, XeLaTeX, or LuaTeX
\usepackage{ifxetex,ifluatex}
\if\ifxetex T\else\ifluatex T\else F\fi\fi T%
  \usepackage{fontspec}
\else
  \usepackage[T1]{fontenc}
  \usepackage[utf8]{inputenc}
  \usepackage{lmodern}
\fi

\usepackage{amsmath}
\usepackage{amssymb}
\usepackage{amsthm}

\usepackage{hyperref}
\usepackage{graphicx}

\title{MA320 (Clontz) Writing Portfolio}
\author{Your name here}


\begin{document}
\maketitle

\section{Writing Samples}
\textit{Include four examples of questions/answers
(not comments) you've posted to
\href{https://math.stackexchange.com/}
{Math.StackExchange} throughout the semester. For each, include
a citation to the post like \cite{1089984},
and be sure to make any necessary adjustments to
typeset your work in LaTeX. Among these, you must include one
question and one answer. You may answer your own question; these
count as separate examples.}

\textit{These posts should reflect
\href{https://usaonline.southalabama.edu/courses/35295/files/6197333?wrap=1}
{Prof. Su's Guidelines for Good Mathematical Writing}
and adhere to Math.SE's policies for
\href{https://math.stackexchange.com/help/asking}
{Asking}
and
\href{https://math.stackexchange.com/help/answering}
{Answering}. Be sure to have at least one example of mathematics
expressed as \LaTeX{} in each sample.}

\textit{For each sample, also provide a few sentences reflecting on it.
This should consider any scores/comments/answers received on your post,
but there is no specific requirement that your posts have any particular
score. Ultimately, was your question well-posed, or your answer useful?}

\subsection{Q/A}
TODO \cite{1089984}

\subsubsection*{Reflection}
TODO

\subsection{Q/A}
TODO \cite{1089984}

\subsubsection*{Reflection}
TODO

\subsection{Q/A}
TODO \cite{1089984}

\subsubsection*{Reflection}
TODO

\subsection{Q/A}
TODO \cite{1089984}

\subsubsection*{Reflection}
TODO


\section{Rubric}
\textit{Self-report your progress by marking the box next to
each of the following requirements that you've [M] met
for this portfolio or [S] show signficiant progress towards.}

\begin{itemize}
\item[[ ]] \textbf{Scope.} This portfolio includes exactly
four posts, including at least one question and at least one
answer.
\item[[ ]] \textbf{Mathematics.} This portfolio engages in
mathematical concepts appropriate for a student enrolled in
MA320. In particular, these posts are generally concerned
with understanding and proving mathematical facts, not just
computing values.
\item[[ ]] \textbf{Reflection.} This portfolio includes
reflections for each post. Each post
is eiether a well-posed question or useful answer.
\item[[ ]] \LaTeX{}. This portfolio was properly typeset
using \LaTeX{}. In particular, mathmode (\verb|\(x^2\)|) was
used appropriately for mathematical expressions (\(x^2\))
in each post, and BibTeX was used for reference management.
\end{itemize}


\section{Overall Quality}
\textit{Self-report your progress by marking the box [X] next to
which of the following best describes the overall quality of
this portfolio.}

\begin{itemize}
\item[[ ]] \textbf{Shows Progress}: This portfolio demonstrates
significant progress towards each of the rubric items.

\item[[ ]] \textbf{Meets Expectations}: This portfolio meets
each of the rubric items.

\item[[ ]] \textbf{Exceeds Expectations}: \textit{If your
portfolio is submitted by the early submission date indicated in Canvas
and accepted by the Instructor as Meets Expectations,
you may discuss with the instructor an appropriate scope of
work to earn this credential.}

\end{itemize}


\bibliographystyle{plain}
\bibliography{references}

\end{document}
