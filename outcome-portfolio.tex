\documentclass{article}

% set font encoding for PDFLaTeX, XeLaTeX, or LuaTeX
\usepackage{ifxetex,ifluatex}
\if\ifxetex T\else\ifluatex T\else F\fi\fi T%
  \usepackage{fontspec}
\else
  \usepackage[T1]{fontenc}
  \usepackage[utf8]{inputenc}
  \usepackage{lmodern}
\fi

\usepackage{amsmath}
\usepackage{amssymb}
\usepackage{amsthm}

\usepackage{hyperref}
\usepackage{graphicx}

\parskip=0.5em

\title{MA320 (Clontz) Outcome Portfolio}
\author{Your name here}


\begin{document}
\maketitle

\section{Quiz Progress}
\textit{For each learning outcome, describe your progress on
its quiz. If you have earned or expect to earn a \checkmark for the
based on existing submissions, be sure to include the \checkmark symbol.
See the comments in the LaTeX source for examples.}

% Examples:
% \begin{itemize}
% \item AB1: \checkmark
% \item CD2: \checkmark expected for my ungraded late quiz
% \item EF3: don't plan to take this quiz
% \item GH4: may take the quiz next week
% \item IJ5: \checkmark expected from my reflection
% \end{itemize}

\begin{itemize}
\item LO1:
\item LO2:
\item LO3:
\item LO4:
\item LO5:
\item ST1:
\item ST2:
\item ST3:
\item ST4:
\item ST5:
\item PF1:
\item PF2:
\item PF3:
\item PF4:
\item PF5:
\item IN1:
\item IN2:
\item RF1:
\item RF2:
\item RF3:
\end{itemize}

\section{Overall Quality}
\textit{Complete this template and
self-report your progress by marking the box }[X]\textit{ next to
which of the following best describes the overall quality of
this portfolio.}

Totaling the \checkmark{}s above, I expect to have met expectations for
%%% FILL IN BELOW
???
%%% FILL IN ABOVE
of our learning outcomes by the end of the semester.

Therefore, this portfolio:

\begin{itemize}
\item[[ ]] \textbf{Shows Progress}: 10 to 17 \checkmark{}s

\item[[ ]] \textbf{Meets Expectations}: 18 or more \checkmark{}s

\item[[ ]] \textbf{Exceeds Expectations}: all 20 \checkmark{}s,
and this portfolio and all relevant work was submitted by the
early deadline noted in Canvas.

\end{itemize}


% \bibliographystyle{plain}
% \bibliography{references}

\end{document}
